\input{preamble}
\input{format}
\input{commands}

\begin{document}

\begin{Large}
    \textsf{\textbf{AMATH 475 Final}}
    
    % Subtitle%
\end{Large}

\vspace{1ex}

\textsf{\textbf{Ziqi Xv}}  \href{mailto:your.email@hotmail.com}{\texttt{z559xu@uwaterloo.ca}}



\textbf{Final hints}: 
Time: 2.5 hr, Location: M3

3 questions

Q1. (25/100)

Conceptual (least mark out of the three)
Some amount of text to read
A part that is very conceptual: Everythin that you uses is what you've seen in class.
a)prove somthing but no math is needed just need to remember things from class 5 marks
b) part b (2 parts 10 points each)
notion of distance in gr

some calculation might be needed but mostly conceptual

Q2. (35/100)
orbital mechanics
one particular problem in orbital mechanics (check all orbital mechanicals from class)
a. 10
b. 10
c. 15

no christoffel symbol just geodesic equations
metric: swarchild or something similar


Q3. (40/100)
Black holes
4 parts?
calculator maybe needed?

recall computation of Christoffel symbol using metric (Levi-Civita connection) (maybe will be tested?)

maximum amount of time to take to hit the singularity of a Schwarchild black hole \pi G M

open book}\\
\textbf{Gaussian State}:
A Gaussian state is a ground or thermal state of a (bosonic or fermionic) Hamiltonian which is quadratic in the creation and annihiliation operators. Those states are fully characterized by expectation values of quadratic operators, and thus $4N^2$
 parameters for $N$
 fermions or bosons. Gaussian state are the states whose Wigner function is a Gaussian function.\\
 \textbf{Fundamental invariants of the electromagnetic field}:
\url{https://physics.stackexchange.com/questions/87817/fundamental-invariants-of-the-electromagnetic-field}
\begin{problem}{Your title}{problem-label}
This is an example problem taken from \cite{Sakurai2020}:

\begin{enumerate}[(a)]
    \item Prove the following
    \begin{enumerate}[label = (\roman*)]
        \item $\langle p' | x | \alpha \rangle = \im \hbar \pdv{}{p'} \langle p' | \alpha \rangle$.

        \item $\langle \beta | x | \alpha \rangle = \int \dd{p'} \phi_{\beta}^{*} (p') \im \hbar \pdv{}{p'} \phi_{\alpha} (p'),$

        where $\phi_{\alpha}(p') = \langle p' | \alpha \rangle$ and $\phi_{\beta}(p') = \langle p' | \beta \rangle$ are momentum-space wave functions.
    \end{enumerate}

    \item What is the physical significance of 
    \[
    \exp\left(\dfrac{\im x \Xi}{\hbar}\right),
    \]

    where $x$ is the position operator and $\Xi$ is some number with the dimension of momentum? Justify your answer.
\end{enumerate}
\end{problem}

Notice that the partial derivative and integral are smaller when used in a sentence compared with when you're working in a math environment like \verb|\begin{equation} \end{equation}|. If you want to display the full size of such commands in a sentence, you must use the command \verb|\displaystyle{}|, like it's shown here:

\begin{problem}{Your title}{problem-label-2}
This is an example problem taken from \cite{Sakurai2020}:

\begin{enumerate}[(a)]
    \item Prove the following
    \begin{enumerate}[label = (\roman*)]
        \item $\langle p' | x | \alpha \rangle = \im \hbar \displaystyle{\pdv{}{p'} }\langle p' | \alpha \rangle$.

        \item $\langle \beta | x | \alpha \rangle = \displaystyle{\int \dd{p'} \phi_{\beta}^{*} (p') \im \hbar \pdv{}{p'} \phi_{\alpha} (p')}$, 

        \vspace{1ex}

        where $\phi_{\alpha}(p') = \langle p' | \alpha \rangle$ and $\phi_{\beta}(p') = \langle p' | \beta \rangle$ are momentum-space wave functions.
    \end{enumerate}

    \item $\cdots$
\end{enumerate}
\end{problem}

I use the package \texttt{physics} which provides a great variety of commands for common operations and symbols. For instance, instead of typing \verb|\dfrac{\partial x}{\partial t}|, the \texttt{physics} package provides the command \verb|\pdv{x}{t}| which gives the same result. I also defined my own commands, so you can take a look in the \texttt{commands.tex} file if you like. I'd also suggest to create a folder and work each problem in a separate \texttt{.tex} file. I already included such folder in the \texttt{Overleaf} template, but you won't see it if you download the \texttt{Github} template. 

% =================================================

% \newpage

% \vfill

\bibliographystyle{apalike}
\bibliography{references}

\end{document}
